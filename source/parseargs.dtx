% \iffalse meta-comment
%
% Copyright (C) 2008-2014 by Ulrich M. Schwarz
% Copyright (C) 2019      by Frank Mittelbach
% Copyright (C) 2020-     by Yukai Chou
%
% This file may be distributed and/or modified under the conditions of
% the LaTeX Project Public License, version 1.3c.
% The license can be obtained from
% http://www.latex-project.org/lppl/lppl-1-3c.txt
%
% \fi
%
%\iffalse (hide this from DocInput)
%<*parseargs>
%\fi
%
% The main command provided by the package is |\parse|\marg{spec}.
% \emph{spec} consists of groups of commands. Each group should set up the
% command |\@parsecmd| which is then run. The important point is that
% |\@parsecmd| will pick up its arguments from the running text, not from
% the rest of \emph{spec}. When it's done storing the arguments,
% |\@parsecmd| must call |\@parse| to continue with the next element of
% \emph{spec}. The process terminates when we run out of spec.
%
% Helper macros are provided for the three usual argument types: mandatory,
% optional, and flag.
%
%\StopEventually{}
%    \begin{macrocode}

\newtoks\@parsespec
\def\parse@endquark{\parse@endquark}
\newcommand\parse[1]{%
  \@parsespec{#1\parse@endquark}\@parse}

\newcommand\@parse{%
  \edef\p@tmp{\the\@parsespec}%
  \ifx\p@tmp\parse@endquark
    \expandafter\@gobble
  \else
%    \typeout{parsespec remaining: \the\@parsespec}%
    \expandafter\@firstofone
  \fi{%
    \@parsepop
  }%
}
\def\@parsepop{%
  \expandafter\p@rsepop\the\@parsespec\@nil
  \@parsecmd
}
\def\p@rsepop#1#2\@nil{%
  #1%
  \@parsespec{#2}%
}

\newcommand\parseOpt[4]{%
  %\parseOpt{openchar}{closechar}{yes}{no}
%  \typeout{attempting #1#2...}%
  \def\@parsecmd{%
    \@ifnextchar#1{\@@reallyparse}{#4\@parse}%
  }%
  \def\@@reallyparse#1##1#2{%
    #3\@parse
  }%
}

\newcommand\parseMand[1]{%
  %\parseMand{code}
  \def\@parsecmd##1{#1\@parse}%
}

\newcommand\parseFlag[3]{%
  %\parseFlag{flagchar}{yes}{no}
  \def\@parsecmd{%
    \@ifnextchar#1{#2\expandafter\@parse\@gobble}{#3\@parse}%
  }%
}
%    \end{macrocode}
%\iffalse
%</parseargs>
%\fi
